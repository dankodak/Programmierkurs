% Vorlage für ein beamer Dokument
% Artikelklasse mit Schriftgrße 12, Papierformat: DIN A4, neue deutsche Rechtschreibung:
\documentclass[12pt,ngerman,a4paper]{beamer}

% Keine Einrückung am Beginn eines Paragraphen
\setlength{\parindent}{0pt}

% Sprachpacket
\usepackage{babel}

% Die folgenden drei Packete sind für Umlaute wichtig
\usepackage[utf8]{inputenc}
\usepackage[T1]{fontenc}
\usepackage{lmodern}

% Erweiterungen in der mathematischen Umgebung
\usepackage{amsmath,amssymb,amsfonts,amsthm}

% Zum Einbinden von Grafiken
\usepackage{graphicx}

% Zum Einfügen von Links mit \url{...}
%  außerdem werden automatisch Links für Querverweise eingefügt
\usepackage{hyperref}

% Benutzerdefinierte Kommandos / Makros
\newcommand{\R}{\ensuremath{\mathbb{R}}}
\newcommand{\Rpos}{\ensuremath{\mathbb{R}_{\geq 0}}}
\newcommand{\norm}[1]{\left\|{#1}\right\|}
\newcommand{\infnorm}[1]{\norm{#1}_\infty}

% Makro zur Demonstration der Abänderbarkeit
\newcommand{\A}{A} % setze dazu \newcommand{\A}{A} auf \newcommand{\A}{~}

% Titel, Autor und Datum setzen
\title{Aufagbe 3}
\subtitle{des Übungsblattes Nummer 8}
\date{Februar 2018}
\author{Philipp Kurz}

% LaTeX beamer Vorlage
\usetheme{Warsaw}

% Einstellung: Beschriftungen (Tabellen, Abbildungen, etc.) nummerieren
\setbeamertemplate{caption}[numbered]

% Inhalt des Dokuments
\begin{document}

% Titel, Autor und Datum ausgeben
\maketitle

% Inhaltsverzeichnis
\begin{frame}
  \tableofcontents{}
\end{frame}


% Abschnitt für Tabellen
\section{Hier steht eine Tabelle}
% Neue Folie für Tabellen
\begin{frame}{Tabellen erstellen}
So sieht eine Tabelle in Latex aus:

% Eine Tabelle
\begin{table}[h!]
  \centering
  \begin{tabular}{|l||r|}
    \hline
    \textbf{Uberschrift Spalte 1}             & \textbf{Überschrift Spalte 2}\\
    \hline
    Spalte 1, Zeile 1	               & Spalte 2, Zeile 1\\
    \hline
    Spalte 1, Zeile 2               & Spalte 2, Zeile 2\\
    \hline
  \end{tabular}
  \caption{Beispieltabelle}
  \label{tab:Tabellen:HHGGDarsteller}
\end{table}
\end{frame}

% Abschnitt für Querverweise und Auflistungen
\section{Querverweise und Auflistungen}\label{sec:QuerverweiseUndAuflistungen}
% Neue Folie für Querverweise und Auflistungen
\begin{frame}{Querverweise und Auflistungen}
In diesem Abschnitt demonstrieren wir, wie Querverweise und Aufzählungen funktionieren. Dazu listen wir hier die Querverweise aus diesem Dokument in zwei Listen auf:
\begin{itemize}
  \item Tabelle \ref{tab:Tabellen:HHGGDarsteller},
  \item Abbildung~\ref{fig:Makros:Aenderbarkeit:MagicASCII},
  \item Abbildung~\ref{fig:Grafiken:Standardbild}.
\end{itemize}
Hier werden alle Gleichungen noch einmal referenziert:
\begin{enumerate}
  \item Gleichung~\ref{eq:Formeln:Wurzel} oder \eqref{eq:Formeln:Wurzel},
  \item Gleichung~\ref{eq:Formeln:Gauss} oder \eqref{eq:Formeln:Gauss}.
\end{enumerate}
\end{frame}

\end{document}